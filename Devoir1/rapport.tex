\documentclass[11pt]{article}

\usepackage{graphicx}
\usepackage{amsmath} 

\title{LINMA1170 rapport Devoir 1}
\author{Dallemagne Brieuc - NOMA 77122100}
\date{23/02/2024} 

\begin{document}

\maketitle

\section{Questions théoriques}

\subsection{Question 1}

Démonstration : Soit \( X \in \mathbb{R}^{n \times p} \).
On a \( J(X) = \lVert AX - B \rVert^2_F = \text{tr}((AX - B)^T (AX - B)) =
\text{tr}(X^T A^T AX - X^T A^T B - B^T A X^T + B^T B) = \text{tr}(X^T A^T AX - 2X^T A^T B + B^T B) \). En dérivant \( J(X) \) par rapport à \( X \), 
on obtient
\[ \nabla J(X) = 2A^T AX - 2A^T B. \]
Donc \( X \) est solution du problème des moindres carrés si et seulement si \( \nabla J(X) = 0 \), c'est-à-dire \( A^T AX = A^T B \). On a donc bien montré que \( X \) est solution du problème des moindres carrés si l'équation normale \( A^T AX = A^T B \) est vérifiée. 
On montre ensuite que ce système admet toujours au moins une solution et qu'elle est unique si \( \text{rang}(A) = n \). 
\newline
Soit \( X \) une solution du problème des moindres carrés. On a alors \( A^T AX = A^T B \). Si \( \text{rang}(A) = n \), alors \( A \) est de rang maximal, donc \( A \) est de plein rang. Donc \( A \) est inversible. On a alors \( X = (A^T A)^{-1}A^T B \). Donc \( X \) est unique. 

\subsection{Question 2}

Démonstration : On a \( A^T AX = A^T B \). On peut alors écrire \( A = QR \), où \( Q \) est une matrice orthogonale et \( R \) est une matrice triangulaire supérieure.
On a alors \( A^T A = R^T Q^T QR = R^T R \). On a alors \( A^T AX = A^T B \) si et seulement si \( R^T RX = R^T Q^T B \) si et seulement si \( RX = Q^T B \). 
On a alors un système triangulaire supérieur à résoudre. On peut alors résoudre ce système par substitution arrière. 

\subsection{Question 3}



\section{Évaluation de la complexité temporelle}
Description de l'expérience numérique pour évaluer la complexité temporelle de la factorisation QR appliquée à des matrices aléatoires de grande taille.

\section{Illustration graphique de l'approximation}
Illustration graphique de quelques résultats de l'approximation appliquée à des suites de points dans différents cas de figure.

\section{Conditions pour une courbe périodique}
Bonus : Explication des conditions nécessaires pour que la courbe soit périodique, illustrées par des exemples.

\section{Conclusion}
Conclusion du rapport.

% Bibliographie
%\bibliographystyle{plain}
%\bibliography{bibliography} % Nom du fichier .bib contenant les références

\end{document}